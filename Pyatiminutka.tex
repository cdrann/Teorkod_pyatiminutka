\documentclass[a4paper, 12pt]{report}

\usepackage[T2A]{fontenc}
\usepackage[utf8]{inputenc}
\usepackage[english,russian]{babel}
\usepackage{amsmath, amsfonts, amssymb, amsthm, mathtools}
\pagestyle{empty}

\usepackage{xcolor}
\definecolor{mygray}{gray}{0.2}

\begin{document}
\pagecolor{white}
\color{black}
\textbf{Пятиминутка №1}

\begin{enumerate}
\item \textbf{Расстояние Хэмминга} - число несовопадающих координат векторов $x, y \in E_{q}^{n} $, обозначается через $d(x,y) = |\lbrace i \in \lbrace1, ... , n\rbrace : x_i \neq y_i \rbrace| $.

\item \textbf{Вес Хэмминга} - число ненулевых координат вектора $x $, обозначается: $\omega (x) $, $\omega (x) = d(x,0) $.

\item \textbf{Код} - обозначается через $C $ и является подмножеством кодовых слов: $C \subseteq E_{q}^{n} $.

\item \textbf{Параметры кода} - $(n, |C|, d) $, где:
$n $ - длина кода,
$|C| $ - мощность кода,
$d $ - минимальное кодовое расстояние, т.е. минимум расстояний по всем парам кодовых слов из $C $.

\item \textbf{Кодер} - биекция из множества информационных сообщений $M $, $M \subseteq E^{k}$ в множество кодовых слов $C $.

\item \textbf{Принцип макисмума правдоподобия} - пусть для передачи использовался код $C $, если $y $ - полученный вектор, то декодируем его в ближайшее кодовое слово $x\in C$. 

\item \textbf{Число исправлемых ошибок} - пусть $C $ код с кодовым расстоянием $d $, и пусть при передаче кодового слова $x \in C $ возникло не более $\lfloor (d-1)/2 \rfloor $ ошибок, тогда декодер восстановит сообщение. 

\item \textbf{Число обнаруживаемых ошибок} - пусть $C $ код с кодовым расстоянием $d $, и пусть при передаче кодового слова $x \in C $ возникло не менее 1 и не более $(d-1) $ и из канала связи получили вектор $y $. В этом случае кодер может запросить снова передачу данных, так как $y $ - не кодовое слово. То есть код обнаруживает $(d-1) $ ошибок.

\item \textbf{Линейный код} - код $C $ называется линейным, если $C $ является векторным подпространством $E_{q}^{n} $.

\item \textbf{Размерность линейного кода $C $} - число векторов в базисе $C $, обозначается через $k $.

\item \textbf{Параметры линейного кода} - $[n, k, d] $, где $n $ - длина кода, $k $ - размерность, $d $ - минимальное кодовое расстояние.

\item \textbf{Кодовое расстояние линейного кода} - оно равно минимальному весу среди ненулевых кодовых слов.

\item \textbf{Порождающая матрица} - матрица $G_{k\times n} $ строки которой образуют базис $C $, называется порождающей матрицей кода $C $.

\item \textbf{Проверочная матрица} - матрица $H_{n-k\times n} $ имеющая $n-k $ строк и $n $ столбцов называется проверочной, если выполнено $H x^{T} = 0^{n-k}$  $\Leftrightarrow $  $x \in C $.

\item \textbf{Порождающая матрица в каноническом виде} - порождающая матрица $G $ называется заданной в каноническом виде, если $G = [E_{k}|A] $, где $E_{k} $ - единичная матрица.

\item \textbf{Проверочная матрица в канониеском виде} - проверочная матрица $H $ называется заданной в каноническом виде, если $H = [A|E_{n-k}] $, где $E_{n-k} $ - единичная матрица.

\item \textbf{Теорема связывающая порождающую и проверочную матрицы} - если $[E_{k}|A] $ - порождающая матрица в каноническом виде кода $C $, тогда $[-A^{T}|E_{n-k}] $ - являестя проверочной матрицей в каноническом виде кода $C $. Верно и обратное. 

\end{enumerate}


\textbf{Пятиминутка №2}

\begin{enumerate}


\item \textbf{Теорема о столбцах проверочной матрицы}
Пусть $H$ – проверочная матрица линейного кода $C$. Кодовое
расстояние $C$ равно $d$ тогда и только тогда когда любые $d - 1$
столбцов $H$ линейно независимы и существует $d$ линейно
зависимых столбцов.

\textbf{Или кратко:}\\
Пусть $H$ - проверочная матрица линейного кода $C$, тогда\\
$d_{C} = d \Leftrightarrow \forall d-1$ столбцов проверочной матрицы $H$ линейно-независимы, и $\exists d$ линейно зависимых столбцов.


\item \textbf{Замечание 1}

* двоичный код с проверочной матрицей $H$.

Кодовое расстояние $C$ равно $1$ тогда и только тогда когда в
его проверочной матрице $H$ существует нулевой столбец.

\item \textbf{Замечание 2}
Кодовое расстояние $C$ равно $2$ тогда и только тогда когда в
$H$ нет нулевых столбцов и есть пара одинаковых столбцов.

\item \textbf{Замечание 3}
Кодовое расстояние $C$ равно $3$ тогда и только тогда когда в
$H$ нет нулевых столбцов, столбцы попарно различны и есть
столбец равный сумме двух других.

\item \textbf{Код Хэмминга}
Пусть $r \geq 2$. Двоичным кодом Хэмминга (с $r$ проверками на
четность) называется код с проверочной матрицей $H$,
столбцами которой являются все ненулевые векторы длины $r$.
Параметры кода Хэмминга:\\
$n = 2^{r} - 1$ - длина кода;\\
$k = n - r$ - размерность кода;\\
$d = 3$ - минимальное кодовое расстояние(все векторы попарно различны, нет нулевых, сумма двух любых столбцов встречается в матрице.)

\item \textbf{Граница Хэмминга. Теорема}
Пусть $C$ –двоичный код длины $n$ и кодовым расстоянием $d$.
Тогда
\[  |C| \leq \frac {2^n} {{\sum_{i = 0}^{\lfloor(d - 1)/2\rfloor} C_n^i}} \]


\item \textbf{Шаром }  $B(x, j)$ радиуса $j$ c центром в векторе $x$ называется
множество всех векторов, находящихся на расстоянии
Хэмминга не более $j$ от $x$.

\textbf{Или кратко:}\\
\[B(x,j) = \lbrace y\in E_{q}^{n} : d(x,y) \leq j\rbrace\]

\item \textbf{Граница Хэмминга для $q$-значных кодов}
Пусть $C$ –- $q$-значный код длины $n$ и кодовым расстоянием $d$.
Тогда
\[  |C| \leq \frac {q^n} {{\sum_{i = 0}^{\lfloor(d - 1)/2\rfloor} C_n^i (q - 1)^i}} \]

\item
$q$-значный код называется \textbf{совершенным} если его мощность
достигает границы Хэмминга.

\textbf{Или кратко:}\\
$C \subseteq E_{q}^{n}$ - \textbf{совершенный} код, если 
\[|C| = \frac {q^n} {{\sum_{i = 0}^{\lfloor(d - 1)/2\rfloor} C_n^i (q - 1)^i}} \] 

\item 
Двоичный код Хэмминга является совершенным кодом с $d = 3$.

Длина $n = 2^r - 1$

Мощность кода равна $2^{n-r}$

Кодовое расстояние $3$

Граница Хэмминга: $2^{n-r} \leq 2^n/(1 + n) = 2^{n-r}$

\newpage

\item \textbf{Теорема (Граница Синглтона)}
Пусть $C$ – q-значный код с параметрами $n, |C|, d$. Тогда
$log_{q}|C| \leq n - d + 1$.

e.g. $C = {(000),(111)}$


\item \textbf{Полный четновесовой код}
$\{x : x \in E^n. w(x) = 0(mod 2)\}$
Параметры $n, |C| = 2^{n - 1}, d = 2$

\item \textbf{Граница Плоткина}
Пусть – двоичный код длины $n$ с минимальным расстоянием
$d$, и $2d > n$. Тогда справедливо неравенство
\[ |C| \leq 2 \lfloor   d/(2d - n) \rfloor \]

\end{enumerate}

\textbf{Пятиминутка №3}

\begin{enumerate}
\item \textbf{Код Адамара}
Рассмотрим код , столбцы порождающей матрицы $G$ которого состоят из всех ненулевых векторов длины $k$:\\

$G = \begin{pmatrix}
0 & 0 & 1 & 0 & 1 & 1 & 1\\
0 & 1 & 0 & 1 & 0 & 1 & 1\\
1 & 0 & 0 & 1 & 1 & 0 & 1
\end{pmatrix}$, этот код называется \textbf{кодом Адамара}.\\

Его параметры: $[2^{k}-1, k, 2^{k}-1]$

\item \textbf{Утверждение(Код Адамара)}
Код Адамара имеет параметры $[2^{k}-1, k, 2^{k}-1]$ и достигает
границы Плоткина.

\item \textbf{Граница Варшамова - Гилберта}
Пусть $\sum_{i=0}^{d-2} C^{i}_{n-1} < 2^{r}$. Тогда сущесвтует линейный код длины $n, k\geq n-r, d' \geq d$.

\item \textbf{Оптимальный код} Код, имеющий максимальную мощность среди всех кодов той же длинны и кодовым расстоянием называется \textbf{оптимальным}.
\end{enumerate}


\textbf{Пятиминутка №4}

\begin{enumerate}
\item \textbf{Теорема (Конструкция Плоткина)}
Пусть $C$ и $D$ коды одинаковой длины $n$ с кодовым расстоянием $d_{1}$ и $d_{2}$ соответственно. Тогда код $C^{2n} = \lbrace(x, x+y)" x\in C, y\in D\rbrace$ имеет длинну $2n$, мощность $|C|*|D|$, кодовое расстояние $min\lbrace 2d_{1}, d_{2}\rbrace$.

\item \textbf{Расширение кода} Пусть $C$ – двоичный код с кодовым расстоянием $d$, $d$–нечетное. Рассмотрим код:
\[\overline{C} = \lbrace (x, \omega (x)\hspace{0.1cm} mod\hspace{0.1cm} 2):x\in C\rbrace\]
Параметры: $\overline{C}\hspace{0.1cm}:\hspace{0.1cm} \overline{n} = n + 1, \hspace{0.1cm} |\overline{C}| = |C|,\hspace{0.1cm} \overline{d} = d + 1$.

\item \textbf{Выкалывание в коде} Выкалывание кодовых координат представляет собой удаление символа в одной координате во всех кодовых словах. Если исходный код $C$ имел параметры: $(n, |C|, d)$, то код $C'$, полученный выкалыванием из $C$, имеет следующие параметры: $(n - 1, |C'|, d')$, где $|C'| \leq |C|, d - 1 \leq d' \leq d$ (заметим, что $|C| = |C'|$, если $d > 1$).

\item \textbf{Укорочение кода} Укорочение кода состоит в следующем:
\begin{itemize}
\item Выбираем все кодовые слова, у которых $i$-й символ равен $0$ (либо $1$). Как правило, выбирается более мощная часть кодовой матрицы с фиксированной координатой $i$, если таковой нет, у которых символ в координате $i$ равен $0$;
\item Удаляем символы в выбранных словах.
\end{itemize}

Из кода $C$ с параметрами $(n; |C|; d)$ получается
$(n - 1; |C'|; d')$ – код $C'$, где $|C'| \geq |C|=2; d' \geq d$.

\item \textbf{Утверждение (эквивалентность кода Хэмминга)} Всякий линейный совершенный код с кодовым расстоянием $3$ есть код Хэмминга и наоборот.

\item \textbf{Эквивалентные двоичные коды} Двоичные коды $C$ и $D$ длины $n$ называются \textit{эквивалентными}, если существует перестановка координатных позиций $\pi$ и вектор $x \in E^{n}$, такие что
\[x + \pi(C) = D,\]
где $x + \pi(C)$ определяется как следующий код:
\[\lbrace x + (y_{\pi(1)}, ... , y_{\pi(n)}) : y\in C\rbrace\]
Обозначим $(y_{\pi(1)}, ... , y_{\pi(n)})$ через $\pi(y)$.\\
Перестановка $\pi$ и сдвиг на $x$ сохраняют расстояние между любыми словами:
\[d(x + \pi(y), x + \pi(y')) = d(\pi(y), \pi(y')) = d(y,y')\]
Поэтому эквивалентные коды имеют одинаковые параметры.

\item \textbf{Смежный класс по коду} Если $C$ - линейный код длинны $n$, то \textit{смежным классом} по коду $C$ называется: 
\[x + C = \lbrace (x_{1}, ..., x_{n}) + y : y\in C\rbrace\] 
для некторого $x \in E^{n}$.

\item \textbf{Замечание (смежные классы кода Хэмминга)} Коды, эквивалентные коду Хэмминга - все коды Хэмминга (содержащие $0^{n}$) и смежные классы по ним, не содержащие $0^{n}$. 

\item \textbf{Теорема Васильев, 1962} Пусть $C$ - произвольный двоичный совершенный код длинны $n$ с кодовым расстоянием $3$ и $\lambda$ - произвольная функция из кода $C$ в множество $\lbrace 0,1\rbrace$. И пусть $|x| \doteq \omega(x) mod 2$. Множество:
\[V_{C,\lambda} = \lbrace(x + y, |x| + \lambda(y), x): x\in E^{n}, y \in C\rbrace \]
является совершенным двоичным кодом длинны $2\cdot n + 1$ с кодовым расстоянием $3$.

\item \textbf{Следствие из теоремы Васильева} Двоичные совершенные коды, не эквивалентные кодам Хэмминга, существуют для любой длины $n$, $n\geq 15$.

\item \textbf{Теорема, Зиновьев, Леонтьев, Титвайнен, 1973}
Пусть $q = p^m$ тогда всякий  нетривиальный (то есть отличный от всего пространства и имеющий мощность больше 2) совершенный код имеет параметры совпадающие с одним из следующих кодов:

1. q-значный код Хэмминга,

2. Двоичный кода Голея $n = 23, k = 12, d = 7$,

3. Троичный $(q = 3)$ код Голея $n = 11, k = 6, d = 5$.

\item \textbf{Лидер смежного класса} - любой вектор наимешьнего веса в этом классе.

\item \textbf{Утверждение о векторе ошибок} Пусть $y = c + e$ - полученный вектор, $c\in C$. тогда вектор ошибок $e$ принадлежит тому же смежному классу что и $y$.

\end{enumerate}

\end{document}